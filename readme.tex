% Created 2014-12-23 Tue 18:33
\documentclass[9pt,b5paper]{article}
\usepackage{graphicx}
\usepackage{xcolor}
\usepackage{xeCJK}
\setCJKmainfont{SimSun}
\usepackage{longtable}
\usepackage{float}
\usepackage{textcomp}
\usepackage{geometry}
\geometry{left=0cm,right=0cm,top=0cm,bottom=0cm}
\usepackage{multirow}
\usepackage{multicol}
\usepackage{listings}
\usepackage{algorithm}
\usepackage{algorithmic}
\usepackage{latexsym}
\usepackage{natbib}
\usepackage{fancyhdr}
\usepackage[xetex,colorlinks=true,CJKbookmarks=true,linkcolor=blue,urlcolor=blue,menucolor=blue]{hyperref}


\lstset{language=c++,numbers=left,numberstyle=\tiny,basicstyle=\ttfamily\small,tabsize=4,frame=none,escapeinside=``,extendedchars=false,keywordstyle=\color{blue!70},commentstyle=\color{red!55!green!55!blue!55!},rulesepcolor=\color{red!20!green!20!blue!20!}}
\author{Jenny Huang}
\date{\today}
\title{Algorithms Review for Job Interview}
\hypersetup{
  pdfkeywords={},
  pdfsubject={},
  pdfcreator={Emacs 24.3.1 (Org mode 8.2.7c)}}
\begin{document}

\maketitle
\tableofcontents


\section{12/20/2014, Saturday}
\label{sec-1}
\begin{itemize}
\item Website (github), program highlight, and chinese input environment all good now;
\item Will configure Linux Mint Java environment later, prefer emacs;
\item 145/168 done before new season review, begin to work on these questions from today.
\item Just got 4 easiest done: \textbf{149/168}
\begin{itemize}
\item min stack,
\item excel sheet column title,
\item compare version number, and
\item intersection of two linked list,
\end{itemize}
\end{itemize}
\section{12/21/2014, Sunday}
\label{sec-2}
\begin{itemize}
\item Only three got done today: \textbf{152/169}
\begin{itemize}
\item maximum gap
\item fraction to recurring decimal
\item majority element
\end{itemize}
\item Don't feel my mind is clear today at all, will look into job searing instead, hopefully tomorrow I can solve more problems, and slightly complicated ones;
\end{itemize}
\section{12/22/2014, Monday}
\label{sec-3}
\begin{itemize}
\item So far got four done: \textbf{156/169}
\begin{itemize}
\item sort list
\item merge k sorted list
\item trapping rain water
\item recovery binary search tree
\end{itemize}
\item am going to work on the rest 6 tonight, so that hopefully tomorrow I would be able to work on the new 10 questions;
\begin{itemize}
\item \textbf{word ladder II}: spent hours on this one, but got really sick with it! I should have solved my problems gradually, like solve the clone graph to understand graph first before touch this one, but I will get this one done later when I have clear mind.
\item regular expression matching
\item divide two integers
\item clone graph
\item find peak element
\end{itemize}
\end{itemize}
\section{12/23/2014, Tuesday}
\label{sec-4}
\begin{itemize}
\item Meet IPO staff this morning got coming semester plans clear at 8:30am in the morning;
\item Will most probably meet some friend and have dinner together; changed to be \textbf{tomorrow noon}
\item Hopefully by this morning's dirruption meeting staff, I could change back my regular schedule instead of 5am-13:30 day time sleeping, target for tonight fall asleep before 12:00am; fall asleep from 10:00-15:30, seems I will change my schedule back as expected tonight\textasciitilde{}!
\item Will not work on Algorithms for today, but work on it hard tomorrow. I have my confidence that I can figure them out, and do great job summarize the questions during my Java round, no worries!
\item so far Got 2 done: \textbf{158/169}
\begin{itemize}
\item find peak element
\item maximal rectangle
\end{itemize}
\item 
\item 
\end{itemize}

\section{我是一个学生}
\label{sec-5}
\begin{verbatim}
#include <pthread.h>
#include <stdlib.h>
#include <stdio.h>

#define SIZE 8   // Size by SIZE matrices

using namespace std;

int main(int argc, char* argv[]) { // sampel mark for 中文是可以的
    pthread_t* thread;  // pointer to a group of threads
    int i;
    if (argc!=2) {    
        printf("Usage: %s number_of_threads\n",argv[0]);
        exit(-1);
    }
    num_thrd = atoi(argv[1]);
    printf("num_thrd: %d\n", num_thrd);
    init_matrix(A);
    printf("\n");
    init_matrix(B);
    thread = (pthread_t*) malloc(num_thrd*sizeof(pthread_t));

    for (i = 1; i < num_thrd; i++) {    
        //printf("address i: %d\n", i);
        int rc = pthread_create(&thread[i], NULL, multiply, &idx[i]);
        if (rc != 0) {
            perror("Can't create thread");
            free(thread);
            exit(-1);
        }
    }

    // main thread works on slice 0
    // so everybody is busy
    // main thread does everything if threadd number is specified as 1
    //int tmp = 0;
    multiply((void*)(&(idx[0])));

    // main thead waiting for other thread to complete
    for (i = 2; i <= num_thrd; i++)
        pthread_join(thread[i-1], NULL);

    printf("\n\n");
    print_matrix(A);
    printf("\n\n\t       * \n");
    print_matrix(B);
    printf("\n\n\t       = \n");
    print_matrix(C);
    printf("\n\n");

    free(thread);

    return 0;
}
\end{verbatim}
% Emacs 24.3.1 (Org mode 8.2.7c)
\end{document}