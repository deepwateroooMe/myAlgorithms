% Created 2014-12-20 Sat 13:33
\documentclass[9pt,b5paper]{article}
\usepackage{graphicx}
\usepackage{xcolor}
\usepackage{xeCJK}
\setCJKmainfont{SimSun}
\usepackage{longtable}
\usepackage{float}
\usepackage{textcomp}
\usepackage{geometry}
\geometry{left=0cm,right=0cm,top=0cm,bottom=0cm}
\usepackage{multirow}
\usepackage{multicol}
\usepackage{listings}
\usepackage{algorithm}
\usepackage{algorithmic}
\usepackage{latexsym}
\usepackage{natbib}
\usepackage{fancyhdr}
\usepackage[xetex,colorlinks=true,CJKbookmarks=true,linkcolor=blue,urlcolor=blue,menucolor=blue]{hyperref}


\lstset{language=c++,numbers=left,numberstyle=\tiny,basicstyle=\ttfamily\small,tabsize=4,frame=none,escapeinside=``,extendedchars=false,keywordstyle=\color{blue!70},commentstyle=\color{red!55!green!55!blue!55!},rulesepcolor=\color{red!20!green!20!blue!20!}}
\author{Jenny Huang}
\date{\today}
\title{Algorithms Review for Job Interview}
\hypersetup{
  pdfkeywords={},
  pdfsubject={},
  pdfcreator={Emacs 24.3.1 (Org mode 8.2.7c)}}
\begin{document}

\maketitle
\tableofcontents


\section{dfkdjf}
\label{sec-1}
\section{�������}
\label{sec-2}
%\begin{verbatim}
\begin{lstlisting}
#include <pthread.h>
#include <stdlib.h>
#include <stdio.h>

#define SIZE 8   // Size by SIZE matrices

using namespace std;

int num_thrd;   // number of threads

int A[SIZE][SIZE], B[SIZE][SIZE], C[SIZE][SIZE];
int idx[] = {0, 1, 2, 3};

void print_matrix(int m[SIZE][SIZE]) {
    int i, j;
    for (i = 0; i < SIZE; i++) {
        printf("\n\t| ");
        for (j = 0; j < SIZE; j++)
            printf("%2d ", m[i][j]);
        printf("|");
    }
}

void init_matrix(int m[SIZE][SIZE]) {
    int i, j, val = 0;
    for (i = 0; i < SIZE; i++)
        for (j = 0; j < SIZE; j++)
            m[i][j] = val++;
}

// thread function: taking "slice" as its argument
void* multiply(void* slice) {
    int s = *((int*)slice);   // retrive the slice info
    //printf("s value: %d\n", s);
    int from = (s * SIZE)/num_thrd; // note that this 'slicing' works fine
    int to = ((s+1) * SIZE)/num_thrd; // even if SIZE is not divisible by num_thrd
    int i,j,k;

    printf("computing slice %d (from row %d to %d)\n", s, from, to-1);
    for (i = from; i < to; i++) {    
        for (j = 0; j < SIZE; j++) {
            C[i][j] = 0;
            for ( k = 0; k < SIZE; k++)
                C[i][j] += A[i][k]*B[k][j];
        }
    }
    printf("finished slice %d\n\n", s);
    return 0;
}

int main(int argc, char* argv[]) {
    pthread_t* thread;  // pointer to a group of threads
    int i;
    if (argc!=2) {    
        printf("Usage: %s number_of_threads\n",argv[0]);
        exit(-1);
    }
    num_thrd = atoi(argv[1]);
    printf("num_thrd: %d\n", num_thrd);
    init_matrix(A);
    printf("\n");
    init_matrix(B);
    thread = (pthread_t*) malloc(num_thrd*sizeof(pthread_t));

    for (i = 1; i < num_thrd; i++) {    
        //printf("address i: %d\n", i);
        int rc = pthread_create(&thread[i], NULL, multiply, &idx[i]);
        if (rc != 0) {
            perror("Can't create thread");
            free(thread);
            exit(-1);
        }
    }

    // main thread works on slice 0
    // so everybody is busy
    // main thread does everything if threadd number is specified as 1
    //int tmp = 0;
    multiply((void*)(&(idx[0])));

    // main thead waiting for other thread to complete
    for (i = 2; i <= num_thrd; i++)
        pthread_join(thread[i-1], NULL);

    printf("\n\n");
    print_matrix(A);
    printf("\n\n\t       * \n");
    print_matrix(B);
    printf("\n\n\t       = \n");
    print_matrix(C);
    printf("\n\n");

    free(thread);

    return 0;
}
\end{lstlisting}
%\end{verbatim}
% Emacs 24.3.1 (Org mode 8.2.7c)
\end{document}