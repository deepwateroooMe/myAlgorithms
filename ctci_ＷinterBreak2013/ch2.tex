\chapter{Linked Lists}
\small{}

\begin{description}
\item[2.1] Write code to remove duplicates from an unsorted linked list.

FOLLOW UP

How would you solve this problem if a temporary buffer is not allowed?

从一个未排序的链表中移除重复的项. 进一步地,如果不允许使用临时的缓存,你如何解决这个问题?

Solution: 

如果可以使用额外的存储空间,我们就开一个数组来保存一个元素的出现情况。 对于这种情况,最好的解决方法当然是使用哈希表,但令人非常不爽的是C++标准里是没有 哈希表的(java里有)。网上有人用ext下的hash\_map,但毕竟不是C++标准里的, 用起来怪怪的,搞不好换个环境就跑不起来了(像Linux和Windows下使用就不一样)。 所以,一般用一个数组模拟一下就好了。但,这里要注意一个问题, 就是元素的边界,比如链表里存储的是int型变量。那么,如果有负值,这个方法就不奏效 了。而如果元素里的最大值非常大,那么这个数组也要开得很大,而数组中大部分空间是用 不上的,会造成空间的大量浪费。

简言之,如果可以用哈希表,还是用哈希表靠谱。

如下代码遍历一遍链表即可,如果某个元素在数组里记录的是已经出现过, 那么将该元素删除。时间复杂度O(n):
\begin{lstlisting}[language=C++]
void removedulicate(node *head){
    if(head==NULL) return;
    node *p=head, *q=head->next;
    hash[head->data] = true;
    while(q){
        if(hash[q->data]){
            node *t = q;
            p->next = q->next;
            q = p->next;
            delete t;
        }
        else{
            hash[q->data] = true;
            p = q; q = q->next;
        }
    }
}
\end{lstlisting}

如果不允许使用临时的缓存(即不能使用额外的存储空间),那需要两个指针, 当第一个指针指向某个元素时,第二个指针把该元素后面与它相同的元素删除, 时间复杂度O(n2 ),代码如下:
\begin{lstlisting}[language=C++]
void removedulicate1(node *head){
    if(head==NULL) return;
    node *p, *q, *c=head;
    while(c){
        p=c; q=c->next;
        int d = c->data;
        while(q){
            if(q->data==d){
                node *t = q;
                p->next = q->next;
                q = p->next;
                delete t;
            }
            else{
                p = q; q = q->next;
            }
        }
        c = c->next;
    }
}
\end{lstlisting}

完整代码如下:
\lstinputlisting[language=C++]{ch2one.cpp}


\item[2.2] Implement an algorithm to find the nth to last element of a singly linked list.

实现一个算法从一个单链表中返回倒数第n个元素。


Solution: 

这道题的考点在于我们怎么在一个单链表中找到倒数第n个元素? 由于是单链表,所以我们没办法从最后一个元素数起,然后数n个得到答案。 但这种最直观的思路显然是没错的,那我们有没有办法通过别的方式,从最后的元素数起数 n个来得到我们想要的答案呢。

这个次序颠倒的思路可以让我们联想到一种数据结构:栈。

我们如果遍历一遍单链表,将其中的元素压栈,然后再将元素一一出栈。那么, 第n个出栈的元素就是我们想要的。

那我们是不是需要显式地用栈这么一个结构来做这个问题呢?答案是否。看到栈,我们应当 要想到递归,它是一种天然使用栈的方式。所以,第一种解决方案用递归,让栈自己帮我 们从链表的最后一个元素数起。

思路很简单,如果指向链表的指针还未空,就不断递归。当指针为空时开始退递归,这个过 程n不断减1,直接等于1,即可把栈中当前的元素取出。代码如下:

\begin{lstlisting}[language=C++]
node *pp;
int nn;
void findNthToLast1(node *head){
    if(head==NULL) return;
    findNthToLast1(head->next);
    if(nn==1) pp = head;
    --nn;
}
\end{lstlisting}

递归的特点就是理解直接,代码短小。所以,很多递归代码看起来都很漂亮(不包括我这个 哈,还用了两个全局变量,比较不美观)。除了递归,这道题目还有别的方法还做吗? 答案还是有。虽然我们没办法从单链表的最后一个元素往前数,但如果我们维护两个指针, 它们之间的距离为n。然后,我将这两个指针同步地在这个单链表上移动,保持它们的距离 为n不变。那么,当第二个指针指到空时,第一个指针即为所求。很tricky的方法, 将这个问题很漂亮地解决了。代码如下:

\begin{lstlisting}[language=C++]
node* findNthToLast(node *head, int n){
    if(head==NULL || n < 1) return NULL;
    node *p=head, *q=head;
    while(n>0 && q){
        q = q->next;
        --n;
    }
    if(n > 0) return NULL;
    while(q){
        p = p->next;
        q = q->next;
    }
    return p;
}
\end{lstlisting}
没有额外的全局变量,代码看起来就要漂亮一些。:P

完整代码如下:

\lstinputlisting[language=C++]{ch2two.cpp}


\item[2.3] Implement an algorithm to delete a node in the middle of a single linked list, given only access to that node.

EXAMPLE

Input: the node ‘c’ from the linked list a->b->c->d->e

Result: nothing is returned, but the new linked list looks like a->b->d->e

实现一个算法来删除单链表中间的一个结点,只给出指向那个结点的指针。

例子:

输入:指向链表a->b->c->d->e中结点c的指针

结果:不需要返回什么,得到一个新链表:a->b->d->e

Solution: 

这个问题的关键是你只有一个指向要删除结点的指针,如果直接删除它,这条链表就断了。 但你又没办法得到该结点之前结点的指针,是的,它连头结点也不提供。在这种情况下, 你只能另觅他径。重新审视一下这个问题,我们只能获得从c结点开始后的指针, 如果让你删除c结点后的某个结点,那肯定是没问题的。比如删除结点d,那只需要把c 的next指针指向e,然后delete d就ok了。好的,如果我们就删除结点d,我们将得到 a->b->c->e,和目标链表只差了一个结点。怎么办呢?把d的数据给c! 结点结构都是一样的,删除谁都一样,最关键的是结点中的数据,只要我们留下的是数据 a->b->d->e就OK了。

思路已经有了,直接写代码?等等,写代码之前,让我们再简单分析一 下可能出现的各种情况(当然包括边界情况)。如果c指向链表的:1.头结点;2.中间结点。 3.尾结点。4.为空。情况1,2属于正常情况,直接将d的数据给c,c的next指针指向d 的next指向所指结点,删除d就OK。情况4为空,直接返回。情况3比较特殊,如果c 指向尾结点,一般会认为直接删除c就ok了,反正c后面也没有结点了,不用担心链表断开。 可是真的是这样吗?代码告诉我们,直接删除c,指向c的那个结点(比如说b)的next指针 并不会为空。也就是说,如果你打印这个链表,它还是会打印出和原来链表长度一样的链表, 而且最后一个元素为0!

关于这一点,原书CTCI中是这么讲的,这正是面试官希望你指出来的。然后, 你可以做一些特殊处理,比如当c是尾结点时,将它的数据设置为一些特殊字符, 这样在打印时就可以不打印它。当然也可以直接不处理这种情况,原书里的代码就是这么做 的。这里,也直接不处理这种情况。代码如下:

\begin{lstlisting}[language=C++]
bool remove(node *c){
    if(c==NULL || c->next==NULL) return false;
    node *q = c->next;
    c->data = q->data;
    c->next = q->next;
    delete q;
    return true;
}
\end{lstlisting}

\lstinputlisting[language=C++]{ch2thr.cpp}


\item[2.4] You have two numbers represented by a linked list, where each node contains a single digit. The digits are stored in reverse order, such that the 1’s digit is at the head of the list. Write a function that adds the two numbers and returns the sum as a linked list.

EXAMPLE

Input: (3 -> 1 -> 5) + (5 -> 9 -> 2)

Output: 8 -> 0 -> 8

你有两个由单链表表示的数。每个结点代表其中的一位数字。数字的存储是逆序的, 也就是说个位位于链表的表头。写一函数使这两个数相加并返回结果,结果也由链表表示。

例子:(3 -> 1 -> 5), (5 -> 9 -> 2)

输入:8 -> 0 -> 8

Solution:
这道题目并不难,需要注意的有:1.链表为空。2.有进位。3.链表长度不一样。 代码如下:

\lstinputlisting[language=C++]{ch2for.cpp}


\item[2.5] Given a circular linked list, implement an algorithm which returns node at the beginning of the loop.

DEFINITION

Circular linked list: A (corrupt) linked list in which a node’s next pointer points to an earlier node, so as to make a loop in the linked list.

EXAMPLE

input: A -> B -> C -> D -> E -> C [the same C as earlier]

output: C

给定一个循环链表,实现一个算法返回这个环的开始结点。

定义:

循环链表:链表中一个结点的指针指向先前已经出现的结点,导致链表中出现环。

例子:

输入:A -> B -> C -> D -> E -> C [结点C在之前已经出现过]

输出:结点C

Solution: 

http://hawstein.com/posts/2.5.html

关于带环链表的题目,《编程之美》中也有讲过。方法很tricky,设置快慢指针, (快指针速度为2,慢指针速度为1)使它们沿着链表移动,如果这个链表中存在环, 那么快指针最终会追上慢指针而指向同一个结点。接下来的问题是,快指针追上慢指针后, 怎么找到这个环的开始结点? 现在我们还没有答案,那让我们先来分析一下,快指针会在哪里追上慢指针。

设环的开始结点(图中的D)前有k个结点,环有n个结点(上图中n从D到K共8个结点)。 快指针fast和慢指针slow一开始都指向头结点head,它们移动k步可到环的开始结点。 假设慢指针走过m个结点后,快指针追上了它,这时快指针走过了2m个结点。 快指针比慢指针多走过的结点都在环里转圈了,是环中结点数n的整数倍,即:

2m - m = pn --> m = pn, p为正整数

如果头结点是第一个结点的话,那么相遇结点就是第m+1=pn+1个结点(慢指针走了m步)。 当环开始结点为起点1时,相遇结点为第pn+1-k个结点(减去前k个结点)。这pn+1-k 有可能是绕环转了很多圈后的一个数,假设继续走过一些结点,它就绕环刚好q圈, 则从相遇结点需要再经过 qn-(pn+1-k)+1=(q-p)n+k个结点,才能回到环的开始结点(图中结点D)。 由于从相遇结点走回到环的开始结点(图中D)所需要步数一定是小于等于一圈的,因此有

(q-p)n+k <= n 

由于q,p,n,k都是正整数,因此有q-p <= 0,否则上式左边大于右边。 因为q-p <= 0,可以得出上式左边是小于等于k的。即:

(q-p)n+k <= k

如果让快指针在相遇结点继续走,不过这次把速度变成了慢指针一样, 那么它要走(q-p)n+k步到达环开始结点,让慢指针从头结点head开始走, 它要走k步到达环开始结点。最后,它们将在环开始结点处相遇。

这个是怎么得出来的呢?假设快指针走了(q-p)n+k个结点到达环的开始结点,这时, 慢指针也走了(q-p)n+k步,它离环的开始结点还有

k - [ (q-p)n + k ] = (p-q)n (步)

从上面我们知道q-p <= 0,因此p-q >= 0。 这说明慢指针离环开始结点的步数正好是环中结点数的整数倍, 所以当慢指针到达环的开始结点时, 快指针(此时它的速度也是1)刚好在环中转了(p-q)圈,然后和慢指针在环的开始结点处相遇。

代码如下:

\begin{lstlisting}[language=C++]
node* loopstart(node *head){
    if(head==NULL) return NULL;
    node *fast = head, *slow = head;
    while(fast && fast->next){
        fast = fast->next->next;
        slow = slow->next;
        if(fast==slow) break;
    }
    if(!fast || !fast->next) return NULL;
    slow = head;
    while(fast!=slow){
        fast = fast->next;
        slow = slow->next;
    }
    return fast;
}
\end{lstlisting}

这个思路确实很巧很tricky。但,还有没有别的方法呢?更直观更简单的方法。 既然这么问了,当然是有了。:p一个无环的链表,它每个结点的地址都是不一样的。 但如果有环,指针沿着链表移动,那这个指针最终会指向一个已经出现过的地址。 答案是不是已经呼之欲出了。嗯,没错,哈希表!

以地址为哈希表的键值,每出现一个地址,就将该键值对应的实值置为true。 那么当某个键值对应的实值已经为true时,说明这个地址之前已经出现过了, 直接返回它就OK了。

由于C++标准中没有哈希表的操作,我用map进行模拟。不过哈希表的插入和取值操作是O(1) 的时间。而map是由一个RB tree组织,为了维护这个RB tree,插入和取值都会花更多的时 间。

代码如下:

\begin{lstlisting}[language=C++]
map<node*, bool> hash;
node* loopstart1(node *head){
    while(head){
        if(hash[head]) return head;
        else{
            hash[head] = true;
            head = head->next;
        }
    }
    return head;
}
\end{lstlisting}

完整代码如下页:
%\lstinputlisting[language=C++]{ch2fiv.cpp}

\end{description}
